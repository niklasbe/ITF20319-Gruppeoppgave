\documentclass[../doc.tex]{subfiles}
\begin{document}
\section{Bruksssituasjoner}
I utviklingen av et digitalt markedsplasssystem for en oppstartsbedrift som fokuserer på omvisninger og guidede turer, er det viktig å skape realistiske brukssituasjoner som illustrerer hvordan de ulike brukerne vil samhandle med systemet.

En slik situasjon kan involvere Emma, en turist som planlegger sin ferie til Roma. Hun er interessert i historie og ønsker en personlig omvisning i Vatikanmuseene. Emma besøker markedsplassen, filtrerer søket sitt etter historiske turer i Roma og velger en omvisning som tilbys av Marco, en erfaren guide med ekspertise i romersk historie. Systemet tillater Emma å se detaljer om omvisningen, lese tidligere anmeldelser fra andre turister, og bestille og betale for turen direkte på plattformen.

På den andre siden har vi Marco, som bruker systemet for å nå ut til et bredere publikum. Han registrerer seg som en guide, oppretter en profil med sin bakgrunn og spesialområder, og legger inn ulike omvisninger han tilbyr med detaljerte beskrivelser og multimedia-innhold for å tiltrekke seg interesserte turister. Etter hver tur oppfordres Marco til å følge opp turistene for å sikre at de er fornøyde, noe som igjen kan føre til positive anmeldelser og mer forretning.

Videre kan vi forestille oss en administrativ situasjon der oppstartsbedriftens team bruker systemet til å overvåke transaksjoner, løse tvister og oppdatere systemet med nye funksjoner basert på tilbakemeldinger fra brukerne. De kan også bruke data fra systemet til å identifisere trender, som de mest populære turene eller tider på året med høy etterspørsel, for å gi bedre anbefalinger og tilpasse markedsføringsstrategier.

Disse scenariene skisserer ikke bare hvordan systemet vil bli brukt, men også hvordan det skal designes for å møte de spesifikke behovene til brukerne, fra turistenes letthet ved booking til guidenes behov for å markedsføre sine tjenester, og til slutt bedriftens krav til systemadministrasjon.
\end{document}