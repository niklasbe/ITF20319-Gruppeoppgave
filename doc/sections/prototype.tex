\documentclass[../doc]{subfiles}
\begin{document}
\section{Prototype}
\subsection{Introduksjon}
For å illustrere et potensielt utseende og funksjonalitet for vår løsning, har vi utviklet en innledende prototype. Det er essensielt å understreke at prototypen ikke er en fullstendig realisering av alle elementene som er oppført i kravspesifikasjonen. Vi har valgt å konsentrere oss om de mest sentrale og vesentlige kravene og har ivaretatt disse i vår foreløpige versjon. Som et resultat av dette fokuset, inneholder prototypen en rekke begrensninger. Vi har tatt omsorg for å detaljert beskrive disse begrensningene i den følgende delen, for å sikre klarhet og forståelse av prototypens nåværende kapasitet.

\subsection{Begrensninger}
Innenfor rammene av prototypen har vi opprettet en enkel modell som inkluderer bare én bruker, én guide og én administrator. Dette er gjort med det formål å demonstrere og fremvise funksjonaliteten og de unike egenskapene til de forskjellige brukertypene og deres respektive innloggingsprosesser. I den endelige versjonen av vår løsning, vil det bli implementert et fullverdig innloggingssystem, som vil være i tråd med de detaljerte spesifikasjonene og kravene vi har beskrevet tidligere i kravspesifikasjonen.

\subsubsection*{Bruker}
I prototypen har vi bevisst valgt å ikke inkludere personlig informasjon for brukeren, slik som navn, e-postadresse, telefonnummer, betalingsdetaljer og lignende. Dette valget ble tatt for å konsentrere demonstrasjonen på de funksjonelle aspektene av hva en etablert bruker kan utføre innenfor systemet. Denne tilnærmingen tillater oss å rette fokus mot systemets operasjonelle kapabiliteter uten å bli distrahert av håndteringen av personlige data.

\subsubsection*{Guide}    
I vår kravspesifikasjon er det detaljert at guider skal ha muligheten til å opprette, redigere og fjerne turer. Denne funksjonaliteten er integrert i prototypen, men det er imidlertid viktig å bemerke at det i prototypen ikke er implementert et system for å legge til nye guider. Videre er alle turene i systemet for øyeblikket tilknyttet en enkelt guide, noe som er en midlertidig forenkling og ikke representerer det forventede virkelige scenariet der turer vil være administrert av en rekke forskjellige guider.
\subsubsection*{Administrator}
Administratorrollen følger de samme begrensningene som både bruker og guide. For prototypen er det tilstrekkelig men kun én administrator for å demonstrere funksjonaliteten en slik rolle vil kunne ha.

\subsubsection*{Lagring}
I vår innledende utviklingsfase har vi valgt å benytte nettleserens localStorage som et midlertidig lagringssystem for å bevare data. Denne teknikken er praktisk for å demonstrere lagringskonsepter og sikre at brukerdata persisters gjennom flere sesjoner i prototypen. Imidlertid er det viktig å anerkjenne at dette er en begrenset tilnærming, hovedsakelig brukt på grunn av dens enkelhet og umiddelbare tilgjengelighet.

I det endelige produktet vil vi migrere til mer avanserte løsninger som databaser. Databaser tilbyr flere fordeler, inkludert men ikke begrenset til, forbedret sikkerhet, mulighet for samtidige brukerforbindelser, effektiv datahåndtering og skalerbarhet, og avanserte spørringsmuligheter som gir grunnlag for omfattende dataanalyse og rapportering. Ved å velge en database vil vi også kunne tilby en mer sømløs brukeropplevelse med raskere innlastningstider og bedre dataintegritet.
\subsection{Hovedside}
Prototypens hovedside har følgende funksjonalitet og elementer:

\begin{enumerate}
    \item \textbf{Logg inn}
    \begin{itemize}
        \item Tar deg til innloggingssiden der du kan logge deg inn som bruker, guide eller admin.
    \end{itemize}

    \item \textbf{Innloggings-status}
    \begin{itemize}
        \item I det høyre-hjørnet er det et status tekst-felt som angir om du er logget inn, som bruker, guide eller admin.
    \end{itemize}

    \item \textbf{Søkefelt}
    \begin{itemize}
        \item Lar deg søke etter turer.
    \end{itemize}

    \item \textbf{Turliste}
    \begin{itemize}
        \item En liste over turer med informasjon. Ved klikk er du sendt til tur-siden.
    \end{itemize}
\end{enumerate}

\subsection{Innloggingsside og Profil}
På innloggingssiden har du tre valg:
\begin{enumerate}
    \item \textbf{Bruker}
    \begin{itemize}
        \item Bruker kan se sine bestillinger på sin profil.
        \item Bruker har mulighet til å bestille turer på tur-sider.
        \item Bruker kan kontakte tur-guide på tur-sider. (Stub.)
    \end{itemize}
    \item \textbf{Guide}
    \begin{itemize}
        \item Guide kan lage en ny tur på sin profil-side.
        \item Guide kan redigere turer på tur-sider.
        \item Guide kan slette turer på tur-sider.
    \end{itemize}
    \item \textbf{Admin}
    \begin{itemize}
        \item Admin kan slette turer på tur-sider.
        \item Admin kan kontakte tur-guide på tur-sider. (Stub.)
    \end{itemize}
\end{enumerate}

\end{document}