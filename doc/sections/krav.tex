\documentclass[../doc.tex]{subfiles}
\begin{document}
\section{Krav}
\subsection{Funksjonelle Krav}
\begin{enumerate}
    \item \textbf{Søke- og Filtreringsfunksjoner (M)}: \\
    Turister skal være i stand til å søke og filtrere turer. Medium på grunn av utviklingen av søkealgoritmer og filtreringslogikk.
    \begin{itemize}
        \item Turisten skal kunne identifisere søkefeltet umiddelbart.
        \item Turisten skal kunne bruke filtreringsalternativer.
    \end{itemize}

    \item \textbf{Profiladministrasjon (M)}: \\
    Guider skal være i stand til å opprette og administrere sine profiler. Medium på grunn av behovet for brukerregistrering og profilhåndtering.
    \begin{itemize}
        \item Guiden skal kunne registrere seg og logge inn.
        \item Guiden skal kunne legge til og redigere informasjon.
    \end{itemize}

    \item \textbf{Booking og Betalingssystem (L)}: \\
    Brukere skal være i stand til å bestille og betale for turer. Stor oppgave på grunn av integrering av betalingsbehandling og bookinglogikk.
    \begin{itemize}
        \item Turisten skal kunne velge en tur og gå til betaling.
        \item Turisten skal kunne fullføre betalingen sikkert.
    \end{itemize}

    \item \textbf{Anmeldelser og Rating (M)}: \\
    Brukere skal være i stand til å lese og skrive anmeldelser. Medium, involverer utvikling av et system for tilbakemeldinger.
    \begin{itemize}
        \item Turisten skal kunne finne og lese anmeldelser.
        \item Turisten skal kunne skrive en anmeldelse.
    \end{itemize}

    \item \textbf{Kommunikasjonsverktøy (M)}: \\
    Turister og guider skal kunne kommunisere effektivt. Medium, krever utvikling av et meldingssystem og varsler.
    \begin{itemize}
        \item Turisten og guiden skal kunne sende og motta meldinger.
        \item Turisten og guiden skal motta varslinger.
    \end{itemize}

    \item \textbf{Administrasjonspanel (L)}: \\
    Administrasjonsteamet skal kunne overvåke transaksjoner. Stort omfang på grunn av behovet for omfattende funksjonalitet for overvåkning og administrasjon.
    \begin{itemize}
        \item Teamet skal kunne se en oversikt over transaksjoner.
        \item Teamet skal kunne gripe inn i tvistesaker.
    \end{itemize}

    \item \textbf{Trendanalyse og Rapportering (L)}: \\
    Administrasjonsteamet skal utføre trendanalyser og rapportering. Stor oppgave fordi det krever avanserte analyser og rapporteringsfunksjoner.
    \begin{itemize}
        \item Teamet skal kunne generere og vise rapporter.
        \item Teamet skal kunne identifisere populære turer.
    \end{itemize}

    \item \textbf{Tilpassede Anbefalinger (L)}: \\
    Turisten skal motta tilpassede turanbefalinger. Stor, krever sofistikerte algoritmer for personalisering.
    \begin{itemize}
        \item Turisten skal få anbefalinger basert på tidligere søk.
    \end{itemize}

    \item \textbf{Multimedieinnhold for Turer (M)}: \\
    Guiden skal kunne laste opp bilder og videoer. Medium på grunn av behovet for en robust opplastings- og visningsfunksjon.
    \begin{itemize}
        \item Guiden skal kunne berike sin turprofil med visuelt innhold.
    \end{itemize}

    \item \textbf{Kalenderintegrasjon (M)}: \\
    Kalenderintegrasjon er klassifisert som medium på grunn av behovet for å integrere med tidsstyringssystemer og brukergrensesnittsinteraksjoner.
    \begin{itemize}
        \item Turisten skal kunne velge datoer for turer.
    \end{itemize}

    \item \textbf{Språkinnstillinger (S)}: \\
    Å tilby språkinnstillinger er en liten oppgave som typisk innebærer å legge til språkvalg og laste tilhørende ressurser.
    \begin{itemize}
        \item Brukeren skal kunne endre språk på brukergrensesnittet.
    \end{itemize}

    \item \textbf{Valutaomregning (S)}: \\
    Valutaomregning er en liten oppgave da det ofte kan implementeres ved hjelp av eksisterende biblioteker eller tjenester.
    \begin{itemize}
        \item Turisten skal kunne se priser i forskjellige valutaer.
    \end{itemize}

    \item \textbf{Interaktivt Kart (M)}: \\
    Implementering av et interaktivt kart er en middels oppgave som krever kartintegrasjon og brukergrensesnittsdesign.
    \begin{itemize}
        \item Turisten skal kunne utforske turer basert på geografisk plassering.
    \end{itemize}

    \item \textbf{Varslinger for Spesialtilbud (M)}: \\
    Å implementere varslingssystemer er en middels oppgave på grunn av behovet for pålitelig og tidskritisk levering av informasjon.
    \begin{itemize}
        \item Turisten skal motta varslinger om spesialtilbud.
    \end{itemize}

    \item \textbf{Favorittliste (S)}: \\
    Favorittliste er en liten funksjon, da den ofte involverer enkel datahåndtering og grensesnittsoppdateringer.
    \begin{itemize}
        \item Turisten skal kunne lagre turer i en favorittliste.
    \end{itemize}

    \item \textbf{Brukerprofil Tilpasning (M)}: \\
    Tilpasning av brukerprofiler er en middels oppgave på grunn av behovet for å håndtere brukerspesifikke data og preferanser.
    \begin{itemize}
        \item Turisten skal kunne tilpasse sin brukerprofil.
    \end{itemize}

    \item \textbf{Automatisk Innsjekking (M)}: \\
    Automatisert innsjekking er en middels oppgave som involverer integrasjon av sanntidssystemer og brukergrensesnitt.
    \begin{itemize}
        \item Guiden skal kunne registrere turistens deltakelse.
    \end{itemize}

    \item \textbf{Integrert Chatbot (M)}: \\
    Implementering av en chatbot anses som en middels oppgave, ettersom den krever utvikling av samtalelogikk og integrasjon med brukergrensesnittet.
    \begin{itemize}
        \item Turisten skal kunne få svar på vanlige spørsmål.
    \end{itemize}

    \item \textbf{Gruppebestillinger (M)}: \\
    Gruppebestillingsfunksjonalitet er en middels oppgave på grunn av kompleksiteten med å håndtere flere brukere og bestillinger samtidig.
    \begin{itemize}
        \item Turisten skal kunne bestille turer for grupper.
    \end{itemize}

    \item \textbf{Gavekort og Kuponger (M)}: \\
    Gavekort og kupongsystemer er en middels oppgave, da de krever sikker håndtering av kampanjekoder og økonomiske transaksjoner.
    \begin{itemize}
        \item Turisten skal kunne kjøpe og innløse gavekort og kuponger.
    \end{itemize}

    \item \textbf{Kundeservice og Support (M)}: \\
    Etablering av et kundeservice- og supportsystem er en middels oppgave som involverer opprettelse av kommunikasjonskanaler og databehandling.
    \begin{itemize}
        \item Brukeren skal kunne kontakte kundeservice.
    \end{itemize}

    \item \textbf{Brukeropplæring og Veiledninger (M)}: \\
    Utarbeidelse av brukeropplæringsmateriell og veiledninger er en middels oppgave, som krever detaljert dokumentasjon og brukervennlig presentasjon.
    \begin{itemize}
        \item Turisten skal ha tilgang til opplæringsressurser.
    \end{itemize}

    \item \textbf{Sikkerhetskopiering og Datarecovery (L)}: \\
    Implementering av sikkerhetskopiering og datarecovery er en stor oppgave på grunn av behovet for robuste og sikre lagringsløsninger.
    \begin{itemize}
        \item Administrasjonsteamet skal kunne gjenopprette tapte data.
    \end{itemize}

    \item \textbf{Tilbakemeldingsmekanisme for Plattformen (S)}: \\
    Å opprette en tilbakemeldingsmekanisme er en liten oppgave, ofte begrenset til brukergrensesnittsintegrering og enkel datainnsamling.
    \begin{itemize}
        \item Brukeren skal kunne gi tilbakemelding på plattformens funksjonalitet.
    \end{itemize}

    \item \textbf{Mobilitetsfunksjoner (M)}: \\
    Utvikling av mobilfunksjoner krever en middels innsats, særlig for å sikre konsistens og funksjonalitet på tvers av enheter.
    \begin{itemize}
        \item Turisten skal kunne bruke plattformen mens de er på farten.
    \end{itemize}

    \item \textbf{Sanntidsvarsling for Guider (M)}: \\
    Sanntidsvarslinger er en middels oppgave, som krever pålitelig og tidskritisk kommunikasjon.
    \begin{itemize}
        \item Guiden skal motta sanntidsvarslinger om nye bookinger og meldinger.
    \end{itemize}

    \item \textbf{Personalisert Dashboard for Brukere (M)}: \\
    Utvikling av et personalisert dashboard er en middels oppgave, som involverer tilpasset datavisning og brukerinteraksjon.
    \begin{itemize}
        \item Turisten skal kunne se en personlig oversikt over sine bookinger og interesser.
    \end{itemize}

    \item \textbf{Automatiske Påminnelser (M)}: \\
    Implementering av automatiske påminnelser er en middels oppgave, som krever integrasjon med kalender- og varslingssystemer.
    \begin{itemize}
        \item Turisten skal motta automatiske påminnelser om kommende turer.
    \end{itemize}

    \item \textbf{Fleksible Betalingsalternativer (M)}: \\
    Å tilby fleksible betalingsalternativer er en middels oppgave, da det krever integrasjon med flere betalingssystemer og brukervalg.
    \begin{itemize}
        \item Turisten skal kunne velge mellom flere betalingsmetoder.
    \end{itemize}

    \item \textbf{Tilgjengelighet for Funksjonshemmede (M)}: \\
    Sikring av plattformens tilgjengelighet for funksjonshemmede er en middels oppgave, som krever nøye design og testing for tilgjengelighetsstandarder.
    \begin{itemize}
        \item Plattformen skal være tilgjengelig for brukere med funksjonshemminger.
    \end{itemize}

    \item \textbf{Sosial Deling (S)}: \\
    Sosial deling anses som en liten oppgave, ettersom det ofte involverer enkel integrering av delingsknapper fra populære sosiale medieplattformer.
    \begin{itemize}
        \item Turisten skal kunne dele turer på sosiale medier.
    \end{itemize}

    \item \textbf{Interaktiv Q\&A Seksjon for hver Tur (M)}: \\
    Implementering av en interaktiv Q\&A-seksjon er en middels oppgave, som krever utvikling av et brukerforum og administrasjon av innlegg.
    \begin{itemize}
        \item Turisten skal kunne stille spørsmål og få svar.
    \end{itemize}

    \item \textbf{Innlogging via Sosiale Medier (M)}: \\
    Integrering av sosial medieinnlogging er en middels oppgave, ettersom det krever sikker autentisering og integrasjon med tredjepartstjenester.
    \begin{itemize}
        \item Brukeren skal kunne logge inn via sosiale medier.
    \end{itemize}

    \item \textbf{Rangeringssystem for Guider (M)}: \\
    Utvikling av et rangeringssystem for guider er en middels oppgave, som involverer innsamling og analyse av brukeranmeldelser.
    \begin{itemize}
        \item Turisten skal kunne se en rangering av guider.
    \end{itemize}

    \item \textbf{Dynamisk Prissetting (L)}: \\
    Implementering av dynamisk prissetting er en stor oppgave, ettersom det krever avanserte algoritmer for å tilpasse priser basert på etterspørsel og brukeradferd.
    \begin{itemize}
        \item Administrasjonsteamet skal kunne sette dynamiske priser.
    \end{itemize}
\end{enumerate}

\subsection{Ikke-funksjonelle Krav}
\begin{enumerate}
    \item \textbf{Skalerbarhet} \\
    Systemet skal kunne håndtere opptil 50 000 samtidige brukere og 5000 samtidige transaksjoner uten ytelsesnedgang.

    \item \textbf{Sikkerhet} \\
    Systemet skal følge beste praksis for datasikkerhet med ende-til-ende-kryptering for transaksjoner og GDPR-overholdelse for databeskyttelse.

    \item \textbf{Tilgjengelighet} \\ 
    Systemet skal være tilgjengelig 99,9\% av tiden, med en maksimal nedetid på 8 timer årlig.

    \item \textbf{Responsivitet} \\
    Plattformen skal tilpasse seg forskjellige skjermstørrelser og enheter, med en lastetid på maksimalt 3 sekunder for hver side.

    \item \textbf{Internasjonalisering} \\
    Plattformen skal støtte minst ti internasjonale språk og de fem mest brukte valutaene for turisme.

    \item \textbf{Ytelse} \\
    Systemet skal kunne laste innhold og svare på brukerforespørsler innen 2 sekunder under normale driftsforhold.

    \item \textbf{Vedlikehold og Support} \\
    Systemet skal støtte hot-swapping av moduler uten nedetid og tilby 24/7 kundestøtte med en maksimal svartid på 4 timer.

    \item \textbf{Modularitet} \\
    Systemarkitekturen skal være modulbasert for å tillate uavhengige oppdateringer og vedlikehold av forskjellige komponenter uten å påvirke andre deler av systemet.

    \item \textbf{Brukervennlighet} \\
    Plattformen skal følge anerkjente prinsipper for UX-design, med brukertester som viser at nye brukere kan utføre grunnleggende oppgaver uten opplæring.

    \item \textbf{Utvidbarhet} \\
    Systemet skal være designet for enkel integrasjon med tredjepartstjenester som betalingsgateways, sosiale medier og andre APIer.

    \item \textbf{Overholdelse av standarder} \\
    Plattformen skal overholde relevante industristandarder og reguleringer, inkludert, men ikke begrenset til, PCI DSS for betalingssikkerhet og ADA for tilgjengelighet.

    \item \textbf{Katastrofegjenoppretting} \\
    Systemet skal ha en robust katastrofegjenopprettingsplan som inkluderer regelmessig sikkerhetskopiering av data og evnen til å raskt gjenopprette operasjoner i tilfelle et systemsvikt.

    \item \textbf{Oppdateringshyppighet} \\
    Plattformen skal støtte kontinuerlig leveranse og integrasjon, med evnen til å rulle ut oppdateringer minst en gang i måneden uten nedetid.

    \item \textbf{Miljøpåvirkning} \\
    Systemet skal ha en grønn IT-strategi som minimerer miljøpåvirkningen, inkludert energieffektiv infrastruktur og støtte for fjernarbeid.
    
    \item \textbf{Dataintegritet} \\
    Systemet skal inneholde mekanismer for å sikre dataintegritet, med transaksjonslogger og regelmessige revisjoner for å oppdage og rette opp feil.
\end{enumerate}
\end{document}