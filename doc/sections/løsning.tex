\documentclass[../doc.tex]{subfiles}
\begin{document}
\section{Løsning}
\subsection*{Introduksjon}
I introduksjonen presenterer vi formålet med systemet og den grunnleggende problemstillingen det søker å løse: å tilrettelegge for effektiv kobling mellom guider og turister, og å forenkle organiseringen og gjennomføringen av omvisninger og guidede turer.

\subsection*{Systemoversikt}
Nedenfor gis en generell beskrivelse av systemet og hovedkomponentene.

\begin{itemize}
    \item 
    En brukervennlig nettside og mobilapplikasjon som fungerer som et grensesnitt for brukerne.
    \item 
    En backend server som behandler data og logikk.
    \item
    En database for lagring av brukerdata, turer, anmeldelser osv.
\end{itemize}

\subsubsection*{Brukerroller og funksjonalitet}
Vi beskriver de forskjellige brukerrollene---turister, guider og administratorer---og hvilke funksjoner som er tilgjengelige for hver rolle. For eksempel, turister kan søke, bestille og betale for turer, mens guider kan opprette og administrere sine omvisninger.

\subsubsection*{Teknologisk plattform}
Uten å gå for dypt inn i tekniske detaljer, forklarer vi plattformens teknologi på et høyt nivå, som at løsningen er bygget på en skybasert arkitektur som sikrer skalerbarhet og tilgjengelighet.

\subsubsection*{Sikkerhet og personvern}
Vi fremhever viktigheten av sikkerhet og personvern i systemet, inkludert sikker lagring av persondata og sikre betalingsløsninger.

\subsubsection*{Skalerbarhet og vedlikehold}
Forklarer hvordan systemet er designet for å håndtere en økende mengde brukere og data over tid, samt planer for vedlikehold og oppdateringer av systemet.

\subsubsection*{Brukeropplevelse (UX/UI Design)}
Beskriver hvordan designet av brukergrensesnittet er brukervennlig, intuitivt og tilgjengelig for alle brukere, med eksempler på skjermbilder eller mock-ups.

\subsubsection*{Integrasjoner}
Informasjon om hvordan systemet kan integreres med andre tjenester, som betalingsportaler eller sosiale medier.

\subsubsection*{Avslutning og Fremtidige Utviklinger}
Avslutter med en oppsummering av systemets fordeler og potensielle fremtidige utvidelser eller funksjoner.
\end{document}